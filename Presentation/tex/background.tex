% Gaurab vai
\section{Why Mapper Is Needed?}

\begin{frame}
	\frametitle{Alignment Needs Mapping}
	
	\begin{itemize}
		\item<1-> Variant Calling
		\begin{itemize}
			\item<1-> Identification of causative genes , candidate genes, passenger and driver genes in many complex diseases, disorders and cancers.
		\end{itemize}
		\item<2-> DNA Binding
		\begin{itemize}
			\item<2-> Finding DNA-Binding sites on specific reference genome sequence.
		\end{itemize}
		\item<3-> Gene Expression
		\begin{itemize}
			\item<3-> Classification of human tumors, profiling breast cancer, Ontological analysis.
		\end{itemize}
	\end{itemize}
\end{frame}
\begin{comment}
	
	\begin{frame}
	\frametitle{Variant Calling}
	Variant calling is a very significant procedure for WES or WGS sequencing and for some
	experiments also for RNA sequence.
	\begin{itemize}
	\item<1-> Identification of causative genes in Mendelian disorders (germline mutations)
	\item<2-> Identification of candidate genes in complex diseases
	\item<3-> Identification of constitutional mutations as well as driver and passenger genes in cancer
	\end{itemize}
	\end{frame}
	
	
	\begin{frame}
	\frametitle{DNA Binding}
	DNA-binding proteins are a specific type of proteins mainly composed of DNA-binding do-mains and thats why have a specific or general inclination for either single or double stranded DNA.
	\begin{itemize}
	\item<1-> That means in many cases those proteins are supposed to bind to a specific site or location of the
	DNA.
	\item<2-> This site of the DNA called a DNA-binding site.
	\end{itemize}
	\end{frame}
	
	\begin{frame}
	\frametitle{Gene Expression}
	For the greater understanding of genetic information measuring gene expression level in any
	particular species or in any particular individual is very important.
	\begin{itemize}
	\item<1-> Classification of human tumors according to the gene level.
	\item<2-> Proper analysis and profiling of breast cancer.
	\item<3-> Ontological analysis for proper biological interpretation for gnomic data and results.
	\item<4-> Proper sub classification of cancer like Myeloid Leukemia .
	\end{itemize}
	\end{frame}
\end{comment}