\documentclass{standalone}
\usepackage{standalone}

\begin{document}
	\chapter{Setting Up NanoMapper}
	A unix based operating system is needed to run this program. Please ensure that the system fulfilled the following requirements:
	\begin{itemize}
		\item The latest version of of C++ compiler. C++11 standard or later. clang version must be 3.2 or higher.
		\item A cmake build system.
		\item The operating system must be a 64-bit machine.
		\item zlib library installed.
	\end{itemize}
	\section{API}
	The SDSL-lite\cite{SDSL} API is used to implement FM-index. It could be installed by the following command lines in the terminal:
	\begin{verbatim}
	    $ git clone https://github.com/simongog/sdsl-lite.git
	    $ cd sdsl-lite
	    $ ./install.sh
	\end{verbatim}
	The API could be un-installed after working by the following command having same directory in the terminal:
	\begin{verbatim}
	    $ ./uninstall.sh
	\end{verbatim}
	\section{Installing NanoMapper}
	The following commands in terminal would let you install NanoMapper in your PC:
	\begin{verbatim}
	    $ git clone https://github.com/enamcse/NanoMapper.git && cd NanoMapper
	    $ g++ -o NanoMapper -std=c++11 -O3 -DNDEBUG -I ~/include -L ~/lib 
	      Main_minimizer.cpp -lz -lsdsl -ldivsufsort -ldivsufsort64
	\end{verbatim}
	\section{Running NanoMapper}
	The NanoMapper could be run on your PC giving the parameters like below:
	\begin{verbatim}
	    $ ./NanoMapper <path_of_reference_file> <path_of_reads_file> 
	      <K-mer_length> <window_length> 
	      <output_file_path_for_naive_fm_index_approach>
	      <output_file_path_for_enhanced_fm_index_approach>
	\end{verbatim}
	Example:
	\begin{verbatim}
	    $ ./NanoMapper synthetic_reference_seq.fasta gen_seq.fasta 14 24 
	      loc_syn_out.txt loc_syn_out_1.txt
	\end{verbatim}
	Note that, each command should be given in one line. Here, it is shown in multi-line for simplicity.
\end{document}